\documentclass[a4paper,12pt]{article}
\usepackage[utf8]{inputenc} 
\usepackage[french]{babel}
\usepackage{graphicx} 
\usepackage{lmodern} 
\usepackage{cite}


% infos du document
\title{Rapport de Stage}
\author{Jennifer Rondineau}
 
\begin{document}
 
\setlength{\parindent}{0cm}
\setlength{\parskip}{1ex plus 0.5ex minus 0.2ex}
\newcommand{\hsp}{\hspace{20pt}}
\newcommand{\HRule}{\rule{\linewidth}{0.5mm}}
\begin{titlepage}
  \begin{sffamily}
  \begin{center}

    % Upper part of the page. The '~' is needed because \\
    % only works if a paragraph has started.
    \includegraphics[scale=0.2]{logo.jpg}~\\[1.5cm]

    \textsc{\LARGE Université de Nantes}\\
    \textsc{\LARGE  }\\
    \textsc{\Large UFR Sciences et Techniques}\\[1.5cm]

    % Title
    \HRule \\[0.4cm]
    { \huge \bfseries Rapport de stage\\[0.4cm]
     }
     \begin{center}
%pl20150511 detection de contaminants ... données de sequençahe nouvelle génération
		Détection d'éventuels contaminants dans des données NGS à l'aide d'un programme en C utilisant l'interface de programmation de BWA.
	\end{center}
    \HRule \\[0.4cm]

    % Author and supervisor
    
      \begin{center} \large
       Jennifer \textsc{Rondineau}\\
       Master 1 Bioinformatique / Biostatistique \\
      \end{center}
      \begin{center} \large
		\emph{Maître de stage} : Pierre \textsc{Lindenbaum}\\
		Institut du Thorax\ \\
		\includegraphics[scale=1.5]{institut.jpg}\\
      \end{center}
      \begin{center} \large
		\emph{Tuteur pédagogique} : Christine \textsc{Sinoquet} \\
      \end{center}

    \vfill

    % Bottom of the page
    {\large 16 mars 2015 — 7 mai 2015}

  \end{center}
  \end{sffamily}
\end{titlepage}
 % page de garde

\clearpage % mettre le sommaire sur une nouvelle page
\renewcommand{\contentsname}{Sommaire} 
\tableofcontents{} % afficher l'index
\clearpage

\section{Introduction}

\section{Principe du séquençage à haut débit}

\section{Les différents formats de fichiers}

\subsection{FASTA}
Le format fasta est le format le plus simple pour stocker et lire les séquences. Il est composé d'une première ligne comportant un "$>$" suivit par le nom de la séquence. Les lignes suivantes ne contiennent que l'enchainement des nucléotides. 

Exemple : \\
$>$ SRR070516.21004915 \\
GTAGGGAAAGAGTGTAGATCTCGGGGGTCGCCGTATCATTAAAAA \\
CGCCGGGGAAAGGTATGGAAAGCATTAAAGAGTGTAGATCTCGAA \\
GTAGCGCCGTAGGAAAGAGTGTAAAAGGGGGGTTCATTAAATCTC 


\subsection{FASTQ}

Le format FASTQ est un format texte issu du format fasta. Il comporte 4 lignes par séquence : \\
- Une ligne comportant un @ pour l'identifiant de la séquence, \\
   - la séquence,  \\
   - une ligne débutant par "+" parfois suivi de l’identifiant, \\
   - les scores de qualité associé à chacune des bases. \\\\
Exemple : \\
@SRR070516.21004915/2 \\
GTAGGGAAAGAGTGTAGATCTCGGGGGTCGCCGTATCATTAAAAA \\
+ \\
3CGCGLDLA76@9BA?$>$=BC=$>$A@GLD3@79$>$=0=B9476@$>$= \\

il existe des fichiers "single", et "paired-end". dans le fichier single, les reads vont tous dans le même sens. Alors que dans le fichier paired-end, les reads vont dans les deux sens (forward et reverse), cette information est indiquée dans le nom de la séquence, à la fin "/1" = forward et "/2" = reverse.

Le score de qualité associé à chacune des bases est codé en code ASCII, qui est une échelle de corrélation de symboles, manière de représenter la qualité en un seul symbole.
\\

\subsection{Le fichier de mapping SAM}
\paragraph{}
Le format SAM (Sequence Alignement/Map) est un fichier texte comportant deux parties : \\
- une partie entête optionnelle \\
- une partie alignement\\
L'entête, si elle est présente, doit obligatoirement être placée avant la partie alignement. Chaque ligne de la partie entête commence par le symbole "@", c'est ce qui la différencie de la partie alignement. 

Chaque ligne d'alignement est constituée de 11 colonnes minimum, comportant des informations essentielles sur la qualité de l'alignement.   






Cette étape consiste à placer les paires ou singles du fichier FASTQ sur une séquence de référence, dans notre cas le génome humain. Le logiciel d'alignement bwa place toutes les positions potentielles du forward et du reverse et choisi ensuite la meilleur position possible. C'est à dire que le forward et le reverse doivent se faire face et ils doivent également être séparé d'environ 300 bases  : illustration. 
\subsection{le fichier BAM}
\paragraph{}
Le fichier BAM correspond à la forme compressée du fichier SAM. Le fichier BAM n'est pas lisible et comporte des données binaires. La conversion du format SAM vers le format BAM est possible grâce au logiciel Samtools. 
 \\
\section{Le programme Contalign}
\paragraph{}


\\
\subsection{Les arguments d'entrées du programme}
\paragraph{}

\\
\section{Choix des références de contaminants}

\\
\section{Test du programme sur les données du laboratoires}



\section{Conclusion}
\paragraph{}


\textsc{\large Références :}\\[1.5cm]

 \bibliography{mybib}
 \bibliographystyle{plain}
\end{document} 
