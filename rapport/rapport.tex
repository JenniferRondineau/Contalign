\documentclass[a4paper,12pt]{article}
\usepackage[utf8]{inputenc} 
\usepackage[french]{babel}
\usepackage{graphicx} 
\usepackage{lmodern} 
\usepackage{cite}



% infos du document
\title{Rapport de Stage}
\author{Jennifer Rondineau}
 
\begin{document}
 
\setlength{\parindent}{0cm}
\setlength{\parskip}{1ex plus 0.5ex minus 0.2ex}
\newcommand{\hsp}{\hspace{20pt}}
\newcommand{\HRule}{\rule{\linewidth}{0.5mm}}
\begin{titlepage}
  \begin{sffamily}
  \begin{center}

    % Upper part of the page. The '~' is needed because \\
    % only works if a paragraph has started.
    \includegraphics[scale=0.2]{logo.jpg}~\\[1.5cm]

    \textsc{\LARGE Université de Nantes}\\
    \textsc{\LARGE  }\\
    \textsc{\Large UFR Sciences et Techniques}\\[1.5cm]

    % Title
    \HRule \\[0.4cm]
    { \huge \bfseries Rapport de stage\\[0.4cm]
     }
     \begin{center}
		Détection d'éventuels contaminants dans des données NGS à l'aide d'un programme en C utilisant l'interface de programmation de BWA.
	\end{center}
    \HRule \\[0.4cm]

    % Author and supervisor
    
      \begin{center} \large
       Jennifer \textsc{Rondineau}\\
       Master 1 Bioinformatique / Biostatistique \\
      \end{center}
      \begin{center} \large
		\emph{Maître de stage} : Pierre \textsc{Lindenbaum}\\
		Institut du Thorax\ \\[1.5]
		\includegraphics[scale=1.5]{institut.jpg}\\[1.5]
      \end{center}
      \begin{center} \large
		\emph{Tuteur pédagogique} : Christine \textsc{Sinoquet} \\
      \end{center}

    \vfill

    % Bottom of the page
    {\large 16 mars 2015 — 7 mai 2015}

  \end{center}
  \end{sffamily}
\end{titlepage}
 % page de garde

Titre du stage : Détection d'éventuels contaminants dans des données NGS à l'aide d'un programme en C utilisant l'interface de programmation de BWA. \\

Résumé : \\
Dans le cadre d'analyse de génome / exome en NGS (Nouvelle génération de séquençage à haut débit), notre objectif était de créer un outil en C permettant d'identifier au sein d'un fichier BAM, les reads non mappés sur le génome humain, et tenter de les identifier sur d'éventuels contaminants (organismes etrangers tel que des phages phi-X, E. Coli...). \\
Cet outil, Contalign, prend en entrée des fichiers BAM et aligne les reads non mappés sur le génome humain, sur une deuxième référence contenant divers contaminants possibles. Contalign rend un rapport du nombre de reads non mappés par échantillon et un bilan des reads mappés sur la référence des contaminants. \\
Afin de pouvoir lire et traiter les informations dans les fichiers BAM, nous avons utilisé la librairie Samtools, et afin de pouvoir réaliser l'alignement des reads non mappés sur la référence des contaminants, nous avons utilisé l'aligneur BWA. Contalign est disponible sur le repositoire git suivant : https://github.com/JenniferRondineau/masterm2.git \\\\

Mots clés : NGS, BAM, BWA, samtools, contaminants. \\
\clearpage
\paragraph{}
résumé en anglais 

\clearpage % mettre le sommaire sur une nouvelle page
\renewcommand{\contentsname}{Sommaire} 
\tableofcontents{} % afficher l'index
\clearpage

\section{Introduction}

La masse importante de données générées par le séquençage à haut débit requière la mise en place d'un système de gestion et traitement des données. Ce traitement comporte différentes étapes notamment le nettoyage des données (cleaning data) et l'alignement des séquences sur une séquence de référence. Dans notre cas, la séquence de référence initiale correspond au génome humain.

Le nettoyage des données NGS consiste en la recherche, au sein des séquences, des adapteurs  utilisés lors du séquençage. Cette recherche se base sur des critères de taille et pourcentage d'homologie.

Malgré ce nettoyage, dans le cas d'étude sur le génome humain, suite à l'alignement contre une séquence de référence, il est encore possible de détecter des reads non mappés. Ces reads sont considérés comme étant des séquences probablement non humaines et donc potentiellement des séquences provenant de contaminants. Le but de ce stage fut de déterminer à l'aide d'un outil en C, la nature de ces reads non mappés. 

Ce travail fût constitué de deux grandes parties. Une première partie concernant la réalisation du programme permettant la détection des reads non mappés et l'alignement des séquences sur une nouvelle référence. Ce programme fait appel à deux librairies. La librairie Samtools permettant de lire les fichiers BAM, ainsi que la librairie de BWA permettant de réaliser l'alignement des séquences contre une séquence de référence. La seconde partie du travail fût dédiée à la conception d'une base comportant les séquences nucléotidiques d'éventuels contaminants. Pour cela, nous nous sommes basés sur la littérature et certains travaux déjà existant. 


\section{Principe du séquençage à haut débit}
...

De nombreux génomes humains sont complétement séquencés et disponibles, notamment suite au Projet 1000 Génomes. Afin de pouvoir tester mon programme je me suis basée sur ces données. \\
\section{Les différents formats de fichiers}


\subsection{FASTA}
Le format fasta est le format le plus simple pour stocker et lire les séquences. Il est composé d'une première ligne comportant un "$>$" suivit par le nom de la séquence. Les lignes suivantes ne contiennent que l'enchainement des nucléotides. 

Exemple : \\
$>$ SRR070516.21004915 \\
GTAGGGAAAGAGTGTAGATCTCGGGGGTCGCCGTATCATTAAAAA \\
CGCCGGGGAAAGGTATGGAAAGCATTAAAGAGTGTAGATCTCGAA \\
GTAGCGCCGTAGGAAAGAGTGTAAAAGGGGGGTTCATTAAATCTC 


\subsection{FASTQ}

Le format FASTQ est un format texte issu du format fasta. Il comporte 4 lignes par séquence : \\
- Une ligne comportant un @ pour l'identifiant de la séquence, \\
   - la séquence,  \\
   - une ligne débutant par "+" parfois suivi de l’identifiant, \\
   - les scores de qualité associé à chacune des bases. \\\\
Exemple : \\
@SRR070516.21004915/2 \\
GTAGGGAAAGAGTGTAGATCTCGGGGGTCGCCGTATCATTAAAAA \\
+ \\
3CGCGLDLA76@9BA?$>$=BC=$>$A@GLD3@79$>$=0=B9476@$>$= \\

il existe des fichiers "single", et "paired-end". dans le fichier single, les reads vont tous dans le même sens. Alors que dans le fichier paired-end, les reads vont dans les deux sens (forward et reverse), cette information est indiquée dans le nom de la séquence, à la fin "/1" = forward et "/2" = reverse.

Le score de qualité associé à chacune des bases est codé en code ASCII, qui est une échelle de corrélation de symboles, manière de représenter la qualité en un seul symbole.
\\

\subsection{Le fichier de mapping SAM}
\paragraph{}
Le format SAM (Sequence Alignement/Map) est un fichier texte comportant deux parties : \\
- une partie entête optionnelle \\
- une partie alignement\\
L'entête, si elle est présente, doit obligatoirement être placée avant la partie alignement. Chaque ligne de la partie entête commence par le symbole "@", c'est ce qui la différencie de la partie alignement. 

Chaque ligne d'alignement est constituée de 11 colonnes minimum, comportant des informations essentielles sur la qualité de l'alignement.   






Cette étape consiste à placer les paires ou singles du fichier FASTQ sur une séquence de référence, dans notre cas le génome humain. Le logiciel d'alignement bwa place toutes les positions potentielles du forward et du reverse et choisi ensuite la meilleur position possible. C'est à dire que le forward et le reverse doivent se faire face et ils doivent également être séparé d'environ 300 bases  : illustration. 
\subsection{le fichier BAM}
\paragraph{}
Le fichier BAM correspond à la forme compressée du fichier SAM. Le fichier BAM n'est pas lisible et comporte des données binaires. La conversion du format SAM vers le format BAM est possible grâce au logiciel Samtools. 
 \\
\section{Le programme Contalign}
\paragraph{}

Contalign est un programme en C prenant comme entrée un fichier BAM comportant une entête avec au minimum une ligne @RG donnant des renseignements sur les groupes de reads (Read Group), et donnant en sortie ce même fichier BAM d'entrée inchangé. Ce programme n'affecte en aucun cas le fichier d'origine.  \\
Ce programme dépend de 3 librairies : 
- la librairie Samtools
- la librairie BWA
- la librairie Htslib
En effet, ce programme utilise certaines fonctions et structures appartenant à Samtools et bwa. La librairie Samtools permet de gérer l'ouverture et l'utilisation des données contenues dans les fichiers BAM. Alors que BWA, permet de réaliser l'alignement des reads contre les références de contaminants. 
\\
\subsection{Les arguments d'entrées du programme}
\paragraph{}
Pour son bon fonctionnement, Contalign nécessite certains arguments essentiels, tel que le fichier BAM d'entrée, un nom de fichier pour le rapport final contenant le nombre de reads non mappés par échantillons et les contaminants détectés, ainsi que la référence de contaminants sur laquelle on veut aligner ces reads. \\
Il existe certaines contraintes pour ces arguments : \\
Le fichier BAM d'entrée doit, comme dit précédément, contenir une entête avec au minimum une ligne @RG donnant des renseignements sur les reads groups. Si ce n'est pas le cas, le programme envoit un message d'erreur et "stop". \\
Le fichier fasta comportant les références des contaminants doit être indexé avant d'être utilisé par Contalign. Cette indexation se fait grâce à l'outil BWA index. 
\\
\section{Choix des références de contaminants}
Une étape importante de ce stage fut de déterminer une liste de contaminants potentiels pouvant correspondre aux reads non mappés identifiés.De part nos connaissances concernant les différentes méthodes de séquençage. Une des premières idées de contaminants envisageables fût des adapteurs, primeurs, plasmides et vecteurs utilisés lors des manipulations. NCBI propose une base contenant une centaine de référence de ce genre de contaminant nommée UniVec.  
L'inconvénient majeur de cette base, 
\\
\section{Test du programme sur les données du laboratoires}



\section{Conclusion}
\paragraph{}
 ...~\cite{Johnson}, et ~\cite{Zoran} \\
 

\textsc{\large Références :}\\[1.5cm]

 \bibliography{mybib}
 \bibliographystyle{plain}
\end{document} 
