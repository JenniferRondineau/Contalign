\documentclass[a4paper,french,10pt]{memoir}
\usepackage[utf8]{inputenc} % encodage
\usepackage[french]{babel} % pour le francais
\usepackage{color,xcolor,listings} % utile pour les bouts de code insérés
\usepackage{hyperref} % pour les liens dans le pdf (sommaire clickable par exemple ;)
\usepackage{graphicx} % pour les images
\usepackage{lmodern} % pour les fonts
\usepackage[babel=true]{csquotes} % csquotes va utiliser la langue définie dans babel pour les quotes francaises
 
 
% infos du document
\title{Rapport de Stage}
\author{Jennifer Rondineau}
 
\begin{document}
 
\setlength{\parindent}{0cm}
\setlength{\parskip}{1ex plus 0.5ex minus 0.2ex}
\newcommand{\hsp}{\hspace{20pt}}
\newcommand{\HRule}{\rule{\linewidth}{0.5mm}}
\begin{titlepage}
  \begin{sffamily}
  \begin{center}

    % Upper part of the page. The '~' is needed because \\
    % only works if a paragraph has started.
    \includegraphics[scale=0.2]{logo.jpg}~\\[1.5cm]

    \textsc{\LARGE Université de Nantes}\\
    \textsc{\LARGE  }\\
    \textsc{\Large UFR Sciences et Techniques}\\[1.5cm]

    % Title
    \HRule \\[0.4cm]
    { \huge \bfseries Rapport de stage\\[0.4cm]
     }
     \begin{center}
%pl20150511 detection de contaminants ... données de sequençahe nouvelle génération
		Détection d'éventuels contaminants dans des données NGS à l'aide d'un programme en C utilisant l'interface de programmation de BWA.
	\end{center}
    \HRule \\[0.4cm]

    % Author and supervisor
    
      \begin{center} \large
       Jennifer \textsc{Rondineau}\\
       Master 1 Bioinformatique / Biostatistique \\
      \end{center}
      \begin{center} \large
		\emph{Maître de stage} : Pierre \textsc{Lindenbaum}\\
		Institut du Thorax\ \\
		\includegraphics[scale=1.5]{institut.jpg}\\
      \end{center}
      \begin{center} \large
		\emph{Tuteur pédagogique} : Christine \textsc{Sinoquet} \\
      \end{center}

    \vfill

    % Bottom of the page
    {\large 16 mars 2015 — 7 mai 2015}

  \end{center}
  \end{sffamily}
\end{titlepage}
 % page de garde
\frontmatter % introduction
\chapter{Introduction}
\paragraph{}

 
\clearpage % mettre le sommaire sur une nouvelle page
\setcounter{tocdepth}{1} % définir la profondeur de l'Index, 1 == Partie + Sous Parties
\renewcommand{\contentsname}{Sommaire} % renommer la "Tables des matières" en "Sommaire"
\tableofcontents{} % afficher l'index
% penser a caler les parties dans differents fichiers ;)
 
\mainmatter % partie principale du document
\chapter{1\iere ~partie}
\section{Prem's}
\paragraph{}
...
\section{Sous Partie LoL}
\paragraph{}
...
\paragraph{}
...
 
 
\chapter{2\ieme ~partie}
\section{Sous Partie 1}
\paragraph{}
...
\paragraph{}
...
\section{Sous Partie 2}
\paragraph{}
...
\paragraph{}
Le \LaTeX c'est trop puissant !
\paragraph{}
...
\paragraph{}
...
 
 
\backmatter % fin
\chapter{Conclusion}
\paragraph{}
Blablabla ...
\listoffigures % index des illustrations
\appendix % annexes
\end{document} 
