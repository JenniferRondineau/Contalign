\setlength{\parindent}{0cm}
\setlength{\parskip}{1ex plus 0.5ex minus 0.2ex}
\newcommand{\hsp}{\hspace{20pt}}
\newcommand{\HRule}{\rule{\linewidth}{0.5mm}}
\begin{titlepage}
  \begin{sffamily}
  \begin{center}

    % Upper part of the page. The '~' is needed because \\
    % only works if a paragraph has started.
    \includegraphics[scale=0.2]{logo.jpg}~\\[1.5cm]

    \textsc{\LARGE Université de Nantes}\\
    \textsc{\LARGE  }\\
    \textsc{\Large UFR Sciences et Techniques}\\[1.5cm]

    % Title
    \HRule \\[0.4cm]
    { \huge \bfseries Rapport de stage\\[0.4cm]
     }
     \begin{center}
%pl20150511 detection de contaminants ... données de sequençahe nouvelle génération
		Détection d'éventuels contaminants dans des données NGS à l'aide d'un programme en C utilisant l'interface de programmation de BWA.
	\end{center}
    \HRule \\[0.4cm]

    % Author and supervisor
    
      \begin{center} \large
       Jennifer \textsc{Rondineau}\\
       Master 1 Bioinformatique / Biostatistique \\
      \end{center}
      \begin{center} \large
		\emph{Maître de stage} : Pierre \textsc{Lindenbaum}\\
		Institut du Thorax\ \\
		\includegraphics[scale=1.5]{institut.jpg}\\
      \end{center}
      \begin{center} \large
		\emph{Tuteur pédagogique} : Christine \textsc{Sinoquet} \\
      \end{center}

    \vfill

    % Bottom of the page
    {\large 16 mars 2015 — 7 mai 2015}

  \end{center}
  \end{sffamily}
\end{titlepage}
